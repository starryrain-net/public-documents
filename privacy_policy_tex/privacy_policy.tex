\documentclass[a4j,titlepage]{ltjsarticle}
\usepackage{enumerate,url,hyperref,filemod}
\hypersetup{
	colorlinks=true,
	linkcolor=black,
	urlcolor=black}
\begin{document}

\title{StarryRain プライバシーポリシー}
\author{StarryRain/合同会社よすよす}
\maketitle

\tableofcontents
\clearpage
\renewcommand{\thesection}{第\arabic{section}条}
\section{目的及び基本方針}
本ポリシーは、当事務所が「創作物から感動」を生み出すクリエイターを擁している事務所として、個人情報の適正な取り扱いについて適切な規範を定めることにより、お客様と当事務所の間の信用関係の構築・維持を図り、かつ、個人情報のやり取りの安全と円滑に資することを目的とします。

\section{定義}
本ポリシーにおいて、次の各号に掲げる用語の意義は、それぞれ当該各号に定めるところによるものとします。
\paragraph{個人情報}\mbox{}\\
生存する個人に関する情報であって、当該情報に含まれる氏名、生年月日、その他の記述等により特定の個人を識別できるもの、または、個人識別符号が含まれるもの。
個人情報データベース等個人情報を含む情報の集合物。
\paragraph{外部リンク}\mbox{}\\
サイトへのリンクのうち、外部のサイトへのリンクであるもの。
\paragraph{外部サイト}\mbox{}\\
外部リンク先のウェブサイトであるもの。

\section{個人情報の利用目的}
当事務所は、本サービスにて取得した個人情報を以下の目的で使用します。

\begin{enumerate}
	\item 本サービスの提供のため。
	\item お客様からの各種お問い合わせに回答するため。(本人確認をすることも含みます。)
	\item お客様が利用中のサービスの各種情報の提供・通知等及び当事務所が提供する他サービスのご案内を提供・通知等するため。
	\item お客様一人ひとりの本サービスご利用状況や趣味趣向に適したサービス・広告等を提供・表示等するため。
	\item 当事務所が定める各種規則に違反したお客様や、不正・不当な目的・動機・理由をもって本サービスを利用しようとするお客様の特定を行うことで、そのご利用をお断りするため。
	\item お客様にご自身の登録情報の閲覧・訂正・削除・ご利用状況の閲覧等を行っていただくため。
	\item 有料サービスにおいて、お客様にご利用料金を請求するため。
	\item 上記目的に付随する各種調査・統計分析のため。
	\item その他、上記の利用目的に付随する目的のため。
\end{enumerate}


前項に定める以外の目的をもって、当事務所が個人情報を取得する場合、当事務所は、その取得に先立ち、お客様ご本人あるいは法定代理人等に対し利用目的を通知・連絡・明示等するものとします。

\section{個人情報の収集}
\begin{enumerate}
	\item 当事務所は、個人情報の収集を以下の方法をもって行います。
	      \begin{enumerate}[(1)]
		      \item お客様から直接、個人情報の提供を受ける方法
		      \item お客様が当社サービスを利用する際に、自動的に個人情報を記録する方法
		      \item 第三者から間接的にユーザーの個人情報の提供を受ける方法
		      \item 刊行物やインターネット等で公開された個人情報を取得する方法
	      \end{enumerate}
	\item 当事務所は、その情報単体では個人情報に該当しない属性情報、Cookie、IP アドレス、広告識別子、位置情報・行動履歴といったインターネットの利用にかかるログ情報および実店舗の購買履歴等の個人に関する情報(以下、総称して「インフォマティブ情報」とします。)をお客様または第三者から取得しています。お客様が当事務所サービスの利用にあたり当事務所に個人情報を提供した場合、当事務所は、当該情報と、当該お客様のインフォマティブ情報を紐付ける場合がありますが、この場合には当該インフォマティブ情報も個人情報として取り扱います。
	\item お客様からの個人情報の提供は任意ですが、当事務所サービスにおいてそれぞれ必要となる項目を入力いただかない場合は、当事務所サービスを受けられない場合があります。
\end{enumerate}


\section{利用目的の変更}

当事務所が本ポリシーで定める、お客様から取得した個人情報の利用目的は、当事務所が合理的であると認める場合に限り変更することがあります。


前項に定める場合は、法令に基づく場合を除き、相当の期間をもってお客様本人あるいは法定代理人等に対して通知・連絡・明示等を致します。


\section{当事務所が設定した外部リンクの取り扱い}
本サービス上において設定された外部リンクによって外部サイトを利用する場合においても、本ポリシーは当該外部サイト上で適用されず、当該外部サイトのプライバシーポリシーが適用されることとなります。


前項に定める場合において、お客様と当該外部サイトの間で発生した一切の事項につきましては、お客様の当該外部サイトにアクセスする動機となったのが、当事務所が設定した外部リンクに起因するもの等であったとしても、当事務所は一切の責任を負いかねます。


\section{個人情報の取り扱いの委託}
\begin{enumerate}
	\item 当事務所は、個人情報の管理・処理等を外部業者等に委託する場合は、適正な委託先を選定するとともに、秘密保持契約をあらかじめ書面にて委託先と取り交わし、当該委託先においてお客様の個人情報が適切に管理・処理等されているか、必要かつ適切な監督を行います。
	\item 当事務所は、当事務所サービスに参加している企業・学校・団体等から個人情報取扱業務の全部または一部を委託された場合、受託業務の遂行のために、委託された個人情報を取り扱います。
	\item 当事務所が当事務所サービスに参加している企業・学校・団体等から分析業務の委託を受けた場合、当該企業・学校・団体等が指定したお客様を対象として、当該企業・学校・団体等から受領したお客様の個人情報に当事務所が保有するお客様の属性情報や位置情報・行動履歴といったインターネットの利用にかかるログ情報を付加して集計した分析結果を作成する場合があります。この場合、氏名や住所の情報を除外し性別や年代の属性情報にする等、当該情報のみでは当該企業・学校・団体等が特定の個人を識別することができない情報に加工した上で提供します。
\end{enumerate}


\section{個人情報の第三者利用}
当事務所は、次の各号に該当する場合を除き、事前にお客様の同意を得ることなく、当該個人情報を第三者に提供することはございません。
\begin{enumerate}
\item 人の生命、身体または財産の保護のために必要があると当事務所が認める場合であって、本人の同意を得ることが困難であるとき。
\item 公衆衛生の向上、児童の健全な育成の推進等のために特に必要があると当事務所が認める場合であって、本人の同意を得ることが困難であるとき。
\item 国の機関もしくは地方公共団体またはその委託を受けた者が法令の定める事務を遂行することに対して協力する必要がある場合であって、本人の同意を得ることにより当該事務の遂行に支障を及ぼすおそれがあるとき。
\item 裁判所、検察庁、警察、税務署又はこれらに準じた権限を持つ機関から、個人情報の開示を求められたとき。
\item 予め次の事項を告知・公表し、かつ、当事務所が個人情報保護委員会に届出をしたとき。
\begin{enumerate}
	\item 利用目的に第三者への提供を含むこと
	\item 第三者に提供されるデータの項目
	\item 第三者への提供の手段または方法
	\item 本人の求めに応じて個人情報の第三者への提供を停止すること
	\item 本人の求めを受け付ける方法
\end{enumerate}
\end {enumerate}


前各項の定めにかかわらず、次に掲げる場合には、当該情報の提供先は第三者に該当しないものとします。
\begin{enumerate}
	\item 当事務所が利用目的の達成に必要な範囲内において個人情報の取扱いの全部又は一部を委託する場合、
	\item 合併その他の事由による事業の承継に伴って個人情報が提供される場合。
	\item 個人情報を特定の者との間で共同して利用する場合であって、その旨並びに共同して利用される個人情報の項目、共同して利用する者の範囲、利用する者の利用目的および当該個人情報の管理について責任を有する者の氏名または名称について、予め本人に通知し、または本人が容易に知り得る状態に置いた場合。
\end{enumerate}


\section{個人情報保護のための安全管理措置}
事務所は、個人情報の漏洩、滅失または毀損等を未然に防止し、かつ、個人情報を利用目的の範囲内で最新・完全・正確な内容に保つように努めるため、個人情報の管理・処理に従事するスタッフと別途秘密保持契約締結する、SSL 暗号化通信を行う等の現時点での技術水準に合った適切かつ必要な安全管理措置を講じ、適宜見直しや是正等を行ってまいります。

\section{個人情報の開示・訂正・利用停止又は消去等の請求}
\begin{enumerate}
	\item 当事務所は、当事務所が保有するお客様の個⼈情報について、個人情報の保護に関する法律(平成15年法律第57号、以下「個人情報保護法」といいます。)に基づく開⽰、訂正・追加・削除、利⽤停⽌・消去等(以下、「開示等」といいます。)の請求があった場合、請求された⽅が本⼈であることを確認した後、個⼈情報保護法に従って対応いたします。
	\item 前項の定めにかかわらず、開示等において多額な費用を有する場合、その他当事務所が開示等を行うことが困難であると判断した場合であって、ご本人の権利利益を保護するために必要な代替措置がとれる場合は、当該代替措置をとるものとします。
	\item 第1項に定めに基づいて開示等の請求がされた場合は、当該請求1回につき、事務手数料として1000円をご負担いただきます。
\end{enumerate}

\section{Cookie その他の技術の使用}
\begin{enumerate}
	\item 本サービスでは、Cookie 及びこれに類似する技術を利用することがございます。
	\item お客様が Cookie の受け取りを拒否した場合、本サービスの機能の一部が制限され、または、利用ができなくなる可能性があることについて、あらかじめ同意したものとみなします。
\end{enumerate}


\section{外国にある第三者への個人情報の提供}

当事務所は、以下の場合において外国(本邦の域外にある国または地域をいいます)にある第三者に個人情報を提供することがあります。
\begin{enumerate}[(1)]
	\item 個人情報を第三者に提供する場合
	      個人情報の提供先が外国にある第三者の場合、本ポリシー第 8 条に定められた範囲で個人情報を利用します。
	      提供先の第三者は、当該国の個人情報保護に関する法規制を遵守しています。
	\item 個人情報取扱業務を外部委託する場合
	      当事務所が個人情報取扱業務の全部または一部を外部委託する委託先の中には、外国にある委託先があります。この場合、当事務所は当該委託先に対して必要かつ適切な監督を行います。
	\item 合併その他の事由による事業の承継に伴い個人情報を提供する場合
	      合併その他の事由による事業の承継先が外国にある事業者であることがあります。この場合、当該事業承継の承継前の利用目的の範囲内で個人情報を取り扱います。
\end{enumerate}


\section{事業責任者・個人情報保護管理者}
クリエイター事務所 StarryRain  代表

\section{お問い合わせ}
本ポリシーに関するご意見、お問い合わせ、苦情、ご質問のお申し出等に関しましては、下記の窓口までお願いいたします。
\subparagraph{クリエイター事務所 StarryRain お問い合わせ窓口}
\url{http://starryrain.net/}


\section{本ポリシーの変更手続き}
当事務所は、社会情勢の変動、技術の発展、個人情報保護法等関連法令の改正などの事由に応じ、お客様への事前の周知・予告なしで本ポリシーを変更する場合がございます。

\begin{flushright}
	\par \date{\Filemodtoday{\jobname}} 改定
	\par\date{2021年11月27日} 制定
\end{flushright}
\end{document}
